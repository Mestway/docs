We present a language to specify instances of the syntax-guided synthesis
(\sygustm) problem. An instance of a \sygustm problem specifies:

\begin{enumerate}
\item
  The vocabularies, theories and the base types that are used to specify
  (a) semantic constraints on the function to be synthesized, and (b) the
  definition of the function to be synthesized itself. We refer to these as
  the \emph{input} and \emph{output} logics respectively.
\item
  A finite set of typed functions $f_1, f_2, \ldots, f_n$ that are to be synthesized.
\item
  The syntactic constraints on each function $f_i, i \in [1,n]$ to be
  synthesized. The syntactic constraints are specified using context-free
  grammars $G_i$ which describe the syntactic structure of candidate
  solution for each of these functions.
  The grammar may only involve (compositions of?) function
  symbols and sorts from the specified output logic.
\item
  The semantic constraints that describe the behavior of the functions to
  the synthesized. This is described as a quantifier-free formula $\varphi$
  in the input logic, and may refer to universally quantified variables
  $v_1, v_2, \ldots, v_m$.
\end{enumerate}

The objective then is to find interpretations $e_i$ (in the form of
expression bodies) for each function $f_i$ such that (a) the expression
body belongs to the grammar $G_i$ that is used to syntactically constrain
solutions for $f_i$, and (b) The constraint $\forall v_1, v_2, \ldots,
v_m\ \varphi(f_1, f_2, \ldots, f_n, v_1, v_2, \ldots, v_m)[f_i \mathtt{:=}
  e_i, \ldots, f_n \mathtt{:=} e_n] $ is satisfied. Note that each $e_i$ is
an expression over the arguments to the function $f_i$. [Do we want to be
  more precise here and talk about synthesizing lambdas and talk about beta
  reduction]?

[TODO: Introduce an example from the old document here? Or do we wait until
  the Examples section later to talk about examples]
