\pdfoutput=1

%% LyX 2.1.5 created this file.  For more info, see http://www.lyx.org/.
%% Do not edit unless you really know what you are doing.
\documentclass[english]{article}
\usepackage[T1]{fontenc}
\usepackage[latin9]{inputenc}
\usepackage{babel}
\usepackage{amsmath}
\usepackage{amssymb}
\usepackage{stmaryrd}
\usepackage{xargs}[2008/03/08]
\usepackage[unicode=true,pdfusetitle,
 bookmarks=true,bookmarksnumbered=false,bookmarksopen=false,
 breaklinks=false,pdfborder={0 0 0},backref=false,colorlinks=false]
 {hyperref}

\makeatletter

%%%%%%%%%%%%%%%%%%%%%%%%%%%%%% LyX specific LaTeX commands.
%% For printing a cirumflex inside a formula
\newcommand{\mathcircumflex}[0]{\mbox{\^{}}}


%%%%%%%%%%%%%%%%%%%%%%%%%%%%%% User specified LaTeX commands.
\usepackage{bussproofs}
\usepackage{centernot}
\usepackage{datetime}
\usepackage{filecontents}
\usepackage{fullpage}
\usepackage{lmodern}
\usepackage{microtype}
\usepackage{tikz}
\usepackage{pgfplots}

\usetikzlibrary{automata}
\usetikzlibrary{backgrounds}
\usetikzlibrary{fit}
\usetikzlibrary{positioning}

\makeatother

\usepackage{listings}
\renewcommand{\lstlistingname}{Listing}

\begin{document}

\title{SyGuS Input Language Standard Version 2.0}


\author{Mukund Raghothaman \and Andrew Reynolds \and Abhishek Udupa}

\maketitle
\theoremstyle{definition}
\newtheorem{example}{Example}
\newcommand{\rem}[1]{\textcolor{red}{[#1]}}
\newcommand{\ajr}[1]{\rem{#1 --ajr}}


\newcommand{\syntax}[1]{{\tt #1}}

\newcommand{\thtable}{\mathcal{T}_{t}}
\newcommand{\sigtable}{\Sigma_{t}}


\global\long\def\autobox#1{#1}


\global\long\def\opname#1{\autobox{\operatorname{#1}}}


\global\long\def\dontcare{\_}


\global\long\def\bbracket#1{\left\llbracket #1\right\rrbracket }


\global\long\def\cnot#1{\centernot#1}






\global\long\def\roset#1{\left\{  #1\right\}  }


\global\long\def\ruset#1#2{\roset{#1\;\middle\vert\;#2}}


\global\long\def\union#1#2{#1\cup#2}


\global\long\def\bigunion#1#2{\bigcup_{#1}#2}


\global\long\def\intersection#1#2{#1\cap#2}


\global\long\def\bigintersection#1#2{\bigcap_{#1}#2}


\global\long\def\bigland#1#2{\bigwedge_{#1}#2}


\global\long\def\powerset#1{2^{#1}}




\global\long\def\cart#1#2{#1\times#2}


\global\long\def\tuple#1{\left(#1\right)}


\global\long\def\quoset#1#2{\left.#1\middle/#2\right.}


\global\long\def\equivclass#1{\left[#1\right]}


\global\long\def\refltransclosure#1{#1^{*}}




\newcommandx\funcapphidden[3][usedefault, addprefix=\global, 1=]{#2#1#3}


\global\long\def\funcapplambda#1#2{\funcapphidden[\,]{#1}{#2}}


\global\long\def\funcapptrad#1#2{\funcapphidden{#1}{\tuple{#2}}}


\global\long\def\funccomp#1#2{#1\circ#2}


\global\long\def\arrow#1#2{#1\to#2}


\global\long\def\func#1#2#3{#1:\arrow{#2}{#3}}




\global\long\def\N{\mathbb{N}}


\global\long\def\Z{\mathbb{Z}}


\global\long\def\Q{\mathbb{Q}}


\global\long\def\R{\mathbb{R}}


\global\long\def\D{\mathbb{D}}


\global\long\def\Zmod#1{\quoset{\Z}{#1\Z}}


\global\long\def\vector#1{\mathbf{#1}}


\global\long\def\dotprod#1#2{#1\cdot#2}


\global\long\def\bool{{\tt bool}}


\global\long\def\true{{\tt true}}


\global\long\def\false{{\tt false}}






\global\long\def\strempty{\epsilon}




\global\long\def\kstar#1{#1^{*}}
\global\long\def\koption#1{#1^{?}}
\global\long\def\kstarn#1#2{#1^{#2}}


\global\long\def\kplus#1{#1^{+}}


\global\long\def\paren#1{{\tt (}#1{\tt )}}




\global\long\def\sygus{\left\langle SyGuS\right\rangle }


\global\long\def\cmd{\left\langle Cmd\right\rangle }

\global\long\def\smtcmd{\left\langle SmtCmd\right\rangle }


\global\long\def\setlogiccmd{\left\langle SetLogicCmd\right\rangle }


\global\long\def\sortdefcmd{\left\langle SortDefCmd\right\rangle }


\global\long\def\fundefcmd{\left\langle FunDefCmd\right\rangle }


\global\long\def\vardeclcmd{\left\langle VarDeclCmd\right\rangle }


\global\long\def\fundeclcmd{\left\langle FunDeclCmd\right\rangle }


\global\long\def\synthfuncmd{\left\langle SynthFunCmd\right\rangle }


\global\long\def\constraintcmd{\left\langle ConstraintCmd\right\rangle }


\global\long\def\checksynthcmd{\left\langle CheckSynthCmd\right\rangle }


\global\long\def\setoptscmd{\left\langle SetOptsCmd\right\rangle }


\global\long\def\symbol{\left\langle Symbol\right\rangle }


\global\long\def\quotedliteral{\left\langle QuotedLiteral\right\rangle }


\global\long\def\sortexpr{\left\langle Sort\right\rangle }


\global\long\def\intconst{\left\langle IntConst\right\rangle }


\global\long\def\realconst{\left\langle RealConst\right\rangle }


\global\long\def\boolconst{\left\langle BoolConst\right\rangle }


\global\long\def\bvconst{\left\langle BVConst\right\rangle }

\global\long\def\hexconst{\left\langle HexConst\right\rangle }
\global\long\def\binaryconst{\left\langle BinConst\right\rangle }

\global\long\def\stringconst{\left\langle StringConst\right\rangle }


\global\long\def\sortedvar{\left\langle SortedVar \right\rangle }


\global\long\def\enumconst{\left\langle EnumConst\right\rangle }


\global\long\def\literal{\left\langle Literal\right\rangle }


\global\long\def\term{\left\langle Term\right\rangle }


\global\long\def\gterm{\left\langle GTerm\right\rangle }


\global\long\def\letterm{\left\langle LetTerm\right\rangle }


\global\long\def\letgterm{\left\langle LetGTerm\right\rangle }


\global\long\def\ntdef{\left\langle NTDef\right\rangle }

\global\long\def\grammardef{\left\langle GrammarDef\right\rangle }


% for datatypes
\global\long\def\dtdec{\left\langle DTDec\right\rangle }
\global\long\def\dtconsdec{\left\langle DTConsDec\right\rangle }


\global\long\def\sortdecl{\left\langle SortDecl\right\rangle }

\global\long\def\option{\left\langle Option\right\rangle }


\global\long\def\fundec{\left\langle FunDec\right\rangle }

\global\long\def\fundef{\left\langle FunDef\right\rangle }


\global\long\def\constdeclkwd{{\tt declare\mbox{-}const}}


\global\long\def\dtdeclkwd{{\tt declare\mbox{-}datatype}}


\global\long\def\dtsdeclkwd{{\tt declare\mbox{-}datatypes}}


\global\long\def\recfundefkwd{{\tt define\mbox{-}fun\mbox{-}rec}}


\global\long\def\recfunsdefkwd{{\tt define\mbox{-}funs\mbox{-}rec}}

\global\long\def\setlogickwd{{\tt set\mbox{-}logic}}


\global\long\def\sortdefkwd{{\tt define\mbox{-}sort}}

\global\long\def\sortdeclkwd{{\tt declare\mbox{-}sort}}

\global\long\def\fundefkwd{{\tt define\mbox{-}fun}}


\global\long\def\vardeclkwd{{\tt declare\mbox{-}var}}

\global\long\def\primedvardeclkwd{{\tt declare\mbox{-}primed\mbox{-}var}}


\global\long\def\fundeclkwd{{\tt declare\mbox{-}fun}}


\global\long\def\synthfunkwd{{\tt synth\mbox{-}fun}}

\global\long\def\synthinvkwd{{\tt synth\mbox{-}inv}}


\global\long\def\constraintkwd{{\tt constraint}}
\global\long\def\constraintinvkwd{{\tt inv\mbox{-}constraint}}


\global\long\def\checksynthkwd{{\tt check\mbox{-}synth}}


\global\long\def\setoptkwd{{\tt set\mbox{-}option}}

\global\long\def\setoptskwd{{\tt set\mbox{-}options}}


\global\long\def\bitveckwd{{\tt BitVec}}


\global\long\def\arraykwd{{\tt Array}}


\global\long\def\intkwd{{\tt Int}}


\global\long\def\boolkwd{{\tt Bool}}


\global\long\def\enumkwd{{\tt Enum}}


\global\long\def\realkwd{{\tt Real}}


\global\long\def\constantkwd{{\tt Constant}}


\global\long\def\varkwd{{\tt Variable}}


\global\long\def\inputvarkwd{{\tt InputVariable}}


\global\long\def\localvarkwd{{\tt LocalVariable}}


\global\long\def\letkwd{{\tt let}}


\global\long\def\truekwd{{\tt true}}


\global\long\def\falsekwd{{\tt false}}

\global\long\def\setextensionkwd{{\tt set\mbox{-}extension}}


\section{Introduction}
\label{sec:Introduction}

\ajr{Intro to sygus and example}

\subsection{Differences from~\cite{}}


\section{Syntax}
\label{sec:syntax}

\ajr{General relation to smt2}

\subsection{Comments}


\subsection{Identifiers}


\subsection{Sorts}


\subsection{Terms}


\subsection{Commands}

\begin{figure}[t]
\[
\begin{array}{rcl}
\sygus & ::= & \kstar{\cmd}\\[2ex]
% & | & \kplus{\cmd}\\
\cmd 
 & ::= & \paren{\constraintkwd\mbox{ }\term} \\
 & | & \paren{\checksynthkwd} \\
 & | & \paren{\vardeclkwd\mbox{ }\symbol\mbox{ }\sortexpr}\\ 
 & | & \paren{\constraintinvkwd\mbox{ }\symbol\mbox{ }\symbol\mbox{ }\symbol\mbox{ }\symbol} \\
 & | & \paren{\setextensionkwd\mbox{ }\symbol} \\
 & | & \paren{\synthfunkwd\mbox{ }\symbol\mbox{ }\paren{\kstar{\sortedvar}}\mbox{ }\sortexpr\mbox{ }\koption{\grammardef}}\\
 & | & \paren{\synthinvkwd\mbox{ }\symbol\mbox{ }\paren{\kstar{\sortedvar}}\mbox{ }\koption{\grammardef}} \\
 & | & \smtcmd \\[2ex]
\smtcmd 
 & ::= & \paren{\constdeclkwd\mbox{ }\symbol\mbox{ }\sortexpr} \\
 & | & \paren{\fundeclkwd\mbox{ }\symbol\mbox{ }\paren{\kstar{\sortexpr}}\mbox{ }\sortexpr} \\
 & | & \paren{\dtdeclkwd\mbox{ }\symbol\mbox{}\dtdec} \\
 & | & \paren{\dtsdeclkwd\mbox{ }\paren{\kstarn{\sortdecl}{n+1} }\mbox{ }\paren{\kstarn{\dtdec}{n+1}}} \\
 & | & \paren{\sortdeclkwd\mbox{ }\symbol\mbox{ }\intconst} \\
 & | & \paren{\fundefkwd\mbox{ }\fundef} \\
 & | & \paren{\recfundefkwd\mbox{ }\fundef} \\
 & | & \paren{\recfunsdefkwd\mbox{ }\paren{\kstarn{\fundec}{n+1}}\mbox{ }\paren{\kstarn{\term}{n+1}}} \\
 & | & \paren{\sortdefkwd\mbox{ }\symbol\mbox{ }\sortexpr} \\
 & | & \paren{\setlogickwd\mbox{ }\symbol} \\
 & | & \paren{\setoptkwd\mbox{ }\option} \\[2ex]
 \sortdecl & ::= & \paren{\symbol\mbox{ }\intconst}\\
 \sortedvar & ::= & \paren{\symbol\mbox{ }\sortexpr}\\
 \fundec & ::= & \paren{\symbol\mbox{ }\paren{\kstar{\sortedvar}}\mbox{ }\sortexpr}\\
 \fundef & ::= & \symbol\mbox{ }\paren{\kstar{\sortedvar}}\mbox{ }\sortexpr\mbox{ }\term\\
 \dtdec & ::= & \paren{\kplus{\dtconsdec}} \\
 \dtconsdec & ::= & \paren{\symbol\mbox{ }\kstar{\sortedvar}} \\
 \grammardef & ::= & \paren{\kstarn{\sortedvar}{n+1}}\mbox{ }\paren{\kstarn{\ntdef}{n+1}} \\
 \ntdef & ::= & \paren{\symbol\mbox{ }\paren{ \kplus{\gterm} } } \\[2ex]
 \gterm 
 & ::= &  \paren{\constantkwd\mbox{ }\sortexpr}\\
 & | &\paren{\varkwd\mbox{ }\sortexpr}\\
 & | & \term\\
\end{array}
\]
\caption{
SyGuS version 2.0 commands.
}
\label{fig:commands}
\end{figure}

Figure~\ref{fig:commands} gives the syntax for commands.


\subsection{Examples}


\section{Semantics of Commands}
\label{sec:semantics}

\ajr{Outline structure}

\subsection{Defining Symbols}


\subsection{Describing Synthesis Constraints}


\subsection{Initiating Synthesis and Synthesizer Output}

\bibliographystyle{plain}
\bibliography{references}

\end{document}
