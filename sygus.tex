\pdfoutput=1

%% LyX 2.1.5 created this file.  For more info, see http://www.lyx.org/.
%% Do not edit unless you really know what you are doing.
\documentclass[english]{article}
\usepackage[T1]{fontenc}
\usepackage[latin9]{inputenc}
\usepackage{babel}
\usepackage{amsmath}
\usepackage{amssymb}
\usepackage{stmaryrd}
\usepackage{xargs}[2008/03/08]
\usepackage[unicode=true,pdfusetitle,
 bookmarks=true,bookmarksnumbered=false,bookmarksopen=false,
 breaklinks=false,pdfborder={0 0 0},backref=false,colorlinks=false]
 {hyperref}

\makeatletter

%%%%%%%%%%%%%%%%%%%%%%%%%%%%%% LyX specific LaTeX commands.
%% For printing a cirumflex inside a formula
\newcommand{\mathcircumflex}[0]{\mbox{\^{}}}


%%%%%%%%%%%%%%%%%%%%%%%%%%%%%% User specified LaTeX commands.
\usepackage{bussproofs}
\usepackage{centernot}
\usepackage{datetime}
\usepackage{filecontents}
\usepackage{fullpage}
\usepackage{lmodern}
\usepackage{microtype}
\usepackage{tikz}
\usepackage{pgfplots}
\usepackage{comment}

\usetikzlibrary{automata}
\usetikzlibrary{backgrounds}
\usetikzlibrary{fit}
\usetikzlibrary{positioning}

\makeatother

\usepackage{listings}
\renewcommand{\lstlistingname}{Listing}

\begin{document}

\title{SyGuS Input Language Standard Version 2.0}


\author{Mukund Raghothaman \and Andrew Reynolds \and Abhishek Udupa}

\maketitle
\theoremstyle{definition}
\newtheorem{example}{Example}
\newcommand{\rem}[1]{\textcolor{red}{[#1]}}
\newcommand{\ajr}[1]{\rem{#1 --ajr}}


\newcommand{\syntax}[1]{{\tt #1}}

\newcommand{\thtable}{\mathcal{T}_{t}}
\newcommand{\sigtable}{\Sigma_{t}}


\global\long\def\autobox#1{#1}


\global\long\def\opname#1{\autobox{\operatorname{#1}}}


\global\long\def\dontcare{\_}


\global\long\def\bbracket#1{\left\llbracket #1\right\rrbracket }


\global\long\def\cnot#1{\centernot#1}






\global\long\def\roset#1{\left\{  #1\right\}  }


\global\long\def\ruset#1#2{\roset{#1\;\middle\vert\;#2}}


\global\long\def\union#1#2{#1\cup#2}


\global\long\def\bigunion#1#2{\bigcup_{#1}#2}


\global\long\def\intersection#1#2{#1\cap#2}


\global\long\def\bigintersection#1#2{\bigcap_{#1}#2}


\global\long\def\bigland#1#2{\bigwedge_{#1}#2}


\global\long\def\powerset#1{2^{#1}}




\global\long\def\cart#1#2{#1\times#2}


\global\long\def\tuple#1{\left(#1\right)}


\global\long\def\quoset#1#2{\left.#1\middle/#2\right.}


\global\long\def\equivclass#1{\left[#1\right]}


\global\long\def\refltransclosure#1{#1^{*}}




\newcommandx\funcapphidden[3][usedefault, addprefix=\global, 1=]{#2#1#3}


\global\long\def\funcapplambda#1#2{\funcapphidden[\,]{#1}{#2}}


\global\long\def\funcapptrad#1#2{\funcapphidden{#1}{\tuple{#2}}}


\global\long\def\funccomp#1#2{#1\circ#2}


\global\long\def\arrow#1#2{#1\to#2}


\global\long\def\func#1#2#3{#1:\arrow{#2}{#3}}




\global\long\def\N{\mathbb{N}}


\global\long\def\Z{\mathbb{Z}}


\global\long\def\Q{\mathbb{Q}}


\global\long\def\R{\mathbb{R}}


\global\long\def\D{\mathbb{D}}


\global\long\def\Zmod#1{\quoset{\Z}{#1\Z}}


\global\long\def\vector#1{\mathbf{#1}}


\global\long\def\dotprod#1#2{#1\cdot#2}


\global\long\def\bool{{\tt bool}}


\global\long\def\true{{\tt true}}


\global\long\def\false{{\tt false}}






\global\long\def\strempty{\epsilon}




\global\long\def\kstar#1{#1^{*}}
\global\long\def\koption#1{#1^{?}}
\global\long\def\kstarn#1#2{#1^{#2}}


\global\long\def\kplus#1{#1^{+}}


\global\long\def\paren#1{{\tt (}#1{\tt )}}




\global\long\def\sygus{\left\langle SyGuS\right\rangle }


\global\long\def\cmd{\left\langle Cmd\right\rangle }

\global\long\def\smtcmd{\left\langle SmtCmd\right\rangle }


\global\long\def\setlogiccmd{\left\langle SetLogicCmd\right\rangle }


\global\long\def\sortdefcmd{\left\langle SortDefCmd\right\rangle }


\global\long\def\fundefcmd{\left\langle FunDefCmd\right\rangle }


\global\long\def\vardeclcmd{\left\langle VarDeclCmd\right\rangle }


\global\long\def\fundeclcmd{\left\langle FunDeclCmd\right\rangle }


\global\long\def\synthfuncmd{\left\langle SynthFunCmd\right\rangle }


\global\long\def\constraintcmd{\left\langle ConstraintCmd\right\rangle }


\global\long\def\checksynthcmd{\left\langle CheckSynthCmd\right\rangle }


\global\long\def\setoptscmd{\left\langle SetOptsCmd\right\rangle }


\global\long\def\symbol{\left\langle Symbol\right\rangle }


\global\long\def\quotedliteral{\left\langle QuotedLiteral\right\rangle }


\global\long\def\sortexpr{\left\langle Sort\right\rangle }


\global\long\def\intconst{\left\langle IntConst\right\rangle }


\global\long\def\realconst{\left\langle RealConst\right\rangle }


\global\long\def\boolconst{\left\langle BoolConst\right\rangle }


\global\long\def\bvconst{\left\langle BVConst\right\rangle }

\global\long\def\hexconst{\left\langle HexConst\right\rangle }
\global\long\def\binaryconst{\left\langle BinConst\right\rangle }

\global\long\def\stringconst{\left\langle StringConst\right\rangle }


\global\long\def\sortedvar{\left\langle SortedVar \right\rangle }


\global\long\def\enumconst{\left\langle EnumConst\right\rangle }


\global\long\def\literal{\left\langle Literal\right\rangle }


\global\long\def\term{\left\langle Term\right\rangle }


\global\long\def\gterm{\left\langle GTerm\right\rangle }


\global\long\def\letterm{\left\langle LetTerm\right\rangle }


\global\long\def\letgterm{\left\langle LetGTerm\right\rangle }


\global\long\def\ntdef{\left\langle NTDef\right\rangle }

\global\long\def\grammardef{\left\langle GrammarDef\right\rangle }


% for datatypes
\global\long\def\dtdec{\left\langle DTDec\right\rangle }
\global\long\def\dtconsdec{\left\langle DTConsDec\right\rangle }


\global\long\def\sortdecl{\left\langle SortDecl\right\rangle }

\global\long\def\option{\left\langle Option\right\rangle }


\global\long\def\fundec{\left\langle FunDec\right\rangle }

\global\long\def\fundef{\left\langle FunDef\right\rangle }


\global\long\def\constdeclkwd{{\tt declare\mbox{-}const}}


\global\long\def\dtdeclkwd{{\tt declare\mbox{-}datatype}}


\global\long\def\dtsdeclkwd{{\tt declare\mbox{-}datatypes}}


\global\long\def\recfundefkwd{{\tt define\mbox{-}fun\mbox{-}rec}}


\global\long\def\recfunsdefkwd{{\tt define\mbox{-}funs\mbox{-}rec}}

\global\long\def\setlogickwd{{\tt set\mbox{-}logic}}


\global\long\def\sortdefkwd{{\tt define\mbox{-}sort}}

\global\long\def\sortdeclkwd{{\tt declare\mbox{-}sort}}

\global\long\def\fundefkwd{{\tt define\mbox{-}fun}}


\global\long\def\vardeclkwd{{\tt declare\mbox{-}var}}

\global\long\def\primedvardeclkwd{{\tt declare\mbox{-}primed\mbox{-}var}}


\global\long\def\fundeclkwd{{\tt declare\mbox{-}fun}}


\global\long\def\synthfunkwd{{\tt synth\mbox{-}fun}}

\global\long\def\synthinvkwd{{\tt synth\mbox{-}inv}}


\global\long\def\constraintkwd{{\tt constraint}}
\global\long\def\constraintinvkwd{{\tt inv\mbox{-}constraint}}


\global\long\def\checksynthkwd{{\tt check\mbox{-}synth}}


\global\long\def\setoptkwd{{\tt set\mbox{-}option}}

\global\long\def\setoptskwd{{\tt set\mbox{-}options}}


\global\long\def\bitveckwd{{\tt BitVec}}


\global\long\def\arraykwd{{\tt Array}}


\global\long\def\intkwd{{\tt Int}}


\global\long\def\boolkwd{{\tt Bool}}


\global\long\def\enumkwd{{\tt Enum}}


\global\long\def\realkwd{{\tt Real}}


\global\long\def\constantkwd{{\tt Constant}}


\global\long\def\varkwd{{\tt Variable}}


\global\long\def\inputvarkwd{{\tt InputVariable}}


\global\long\def\localvarkwd{{\tt LocalVariable}}


\global\long\def\letkwd{{\tt let}}


\global\long\def\truekwd{{\tt true}}


\global\long\def\falsekwd{{\tt false}}

\global\long\def\setextensionkwd{{\tt set\mbox{-}extension}}


\section{Introduction}
\label{sec:Introduction}

\ajr{Intro to sygus and example}

\subsection{Differences from Previous Versions}

In this section, we cover the differences
in this format with respect to the one described in the previous
SyGuS standard document~\cite{},
and the extensions described in~\cite{}.

\begin{enumerate}

\item The concrete syntax for the $\synthfunkwd$
was changed so that non-terminal symbols are declared upfront.
%\rem{Mention Start symbol change if applicable.}

\item 
The keyword ${\tt Start}$, which 
denoted the starting (non-terminal) symbol of
grammars in this previous standard, has been removed.
Instead, the first symbol listed in the grammar is assumed
to be the starting symbol.

\item 
Terms that occur as the right hand side of production rules in
SyGuS grammars are now assumed to be binder-free,
in particular, this means that let-terms in this context are now disallowed.
Accordingly,
the keywords $\inputvarkwd$ and $\localvarkwd$,
which were used to specify input and local variables in grammars
respectively, have been removed.

\item The datatype keyword $\enumkwd$
and related syntactic features were removed.
The standard SMT LIB version 2.6 commands
for declaring datatypes are adopted.

\item The $\setoptskwd$ has been renamed $\setoptkwd$ to 
correlate to the existing SMT LIB version 2.6 command.

\item 
The syntax for existing terms and sorts
that was not compliant with SMT LIB versions 2.0 and later have been removed.
This includes two notable exceptions from the previous format.
First,
the syntax for bit-vectors sorts ${\tt (BitVec\ n)}$
is now written  ${\tt (\_\ BitVec\ n)}$.
Second,
all let-bindings do \emph{not} annotate the type of the variable being bound.
Previously, a let-term could be written ${\tt (let\ ((x\ t\ T)) ...)}$
where ${\tt T}$ indicates the type of ${\tt t}$.
Now, it must be written in the SMT-LIB compliant
way ${\tt (let\ ((x\ t)) ...)}$.

%Furthermore, wherever applicable,
%the syntax for sorts and terms
%from other SMT theories assumes the syntax
%prescribed in theory definitions from SMT LIB~\cite{}.

\item The command $\primedvardeclkwd$, which was syntax
sugar for two $\vardeclkwd$ commands in the previous standard, has been removed
for the sake of simplicity.
This command did not provide any benefit to
invariant synthesis problems, since invariant constraints $\constraintinvkwd$
do not accept global variables.
%\rem{TODO: This command is not used properly in current
%invariant synthesis benchmarks (since variables cannot be passed
%to inv-constraint commands), and only contributes
%to confusion, in my opinion. Ok to remove?
%}

%\item Change (fail)?

%\item Elaborations on the well-formedness of all commands.

\item 
The command $\fundeclkwd$,
which declared a universal variable of function type
in the previous standard,
has been removed.
%Solvers that wish to support synthesis conjectures
%with higher-order quantification

\item
A formal notion of features have been added to the background logic,
and the command $\setfeaturekwd$
have been added to the language,
which is used for further refining restrictions or extensions
regarding the kinds of constraints and grammars that may appear in an input.

\end{enumerate}

\section{Syntax}
\label{sec:syntax}

In this section, we describe the concrete syntax
of SyGuS version 2.0 inputs.
This syntax borrows many definitions
from the SMT-LIB version 2.6 format~\cite{}.
In the following description,
italicized text within angle-brackets represents EBNF non-terminals,
and text in typewriter font represents terminal symbols.

A SyGuS input $\sygus$ is thus a sequence of zero or more commands.
\[
\begin{array}{rcl}
\sygus & ::= & \kstar{\cmd}\\[2ex]
\end{array}
\]
The syntax for commands is given at the end of this section.
We first introduce the necessary preliminary definitions.

\subsection{Comments}

Comments in SyGuS specifications are indicated by a semicolon ${\tt ;}$.
After encountering a ${\tt ;}$, the rest of the line is ignored.

\subsection{Literals}
\label{ssec:literals}

A \emph{literal} $\literal$ is a special sequence of characters
used to denote values.
The SyGuS format includes syntax for several kinds of literals,
listed below.

\begin{alignat*}{1}
 & \begin{array}{rcl}
\literal & ::= & \begin{array}{ccccccccccc}
\intconst & | & \realconst & | & \boolconst & | \\
\hexconst & | & \binaryconst & | & \stringconst\end{array}\\
\intconst & ::= & \begin{array}{ccc}
\kplus{\left[{\tt 0}-{\tt 9}\right]} & | & {\tt -}\kplus{\left[{\tt 0}-{\tt 9}\right]}\end{array}\\
\realconst & ::= & \begin{array}{ccc}
\kplus{\left[{\tt 0}-{\tt 9}\right]}{\tt .}\kplus{\left[{\tt 0}-{\tt 9}\right]} & | & {\tt -}\kplus{\left[{\tt 0}-{\tt 9}\right]}{\tt .}\kplus{\left[{\tt 0}-{\tt 9}\right]}\end{array}\\
\boolconst & ::= & \begin{array}{ccc}
\truekwd & | & \falsekwd\end{array}\\
\hexconst & ::= & \begin{array}{c}
{\tt \#x}\kplus{\left(\begin{array}{ccccc}
\left[{\tt 0-9}\right] & | & \left[{\tt a}-{\tt f}\right] & | & \left[{\tt A}-{\tt F}\right]\end{array}\right)}\end{array}\\
\binaryconst & ::= & \begin{array}{c}
{\tt \#b}\kplus{\left[{\tt 0}-{\tt 1}\right]}\end{array}\\
\end{array}
\end{alignat*}

Integer constants are written as usual, in decimal, with an optional
minus at the beginning to denote a negative number. Real numbers are
written using their decimal expansion: at least one decimal digit
before and after a mandatory period, and an optional minus sign at
the beginning. $\truekwd$ and $\falsekwd$ are the predefined boolean
constants.
Hexidecimal and binary constants are written in their
traditional representations.

Additionally,
a string literal $\stringconst$
is any sequence of printable characters
delimited by double quotes ``''.
The characters within these delimiters
are interpreted as denoting characters of the string,
with one exception:
two consecutive double quotes within a string
denotes a single double quotes character.
In other words, ``${\tt a}$``''${\tt b}$'' denotes the string
whose characters are ${\tt a}$, '' and ${\tt b}$.
Notice that 
sequences such as ${\tt \backslash n}$ that are commonly
interpreted as escape sequences are not handled specially,
meaning the above sequence should be interpreted 
as a sequence of two characters ${\tt \backslash}$ and ${\tt n}$.
%\rem{TODO: double quotes within strings escape, no other escape}
This treatment of string literals coincides with
string literals in the SMT LIB version 2.6 format.
For full details, see Section 3.1 of~\cite{}.

Literals are commonly
used for denoting 0-ary symbols of a theory.
For example, 
the theory of integer arithmetic 
uses integer constants to denote integer values.
The theory of bit-vectors uses both
hexidecimal and binary constants in the above syntax
to denote bit-vector vectors.
For more details on example theories,
see Section~\ref{ssec:smt-logic}.

\subsection{Symbols}

Symbols are denoted with the non-terminal $\symbol$. 
A symbol
is any non-empty sequence of upper- and lower-case alphabets, digits,
and certain special characters (listed below), with the restriction that it may not
begin with a digit and is not a reserved word (see Appendix~\ref{apx:reserved} 
for a full list of reserved words).
A special character is any of the following:
\[
{\tt \_}\mbox{ }{\tt +}\mbox{ }{\tt -}\mbox{ }{\tt *}\mbox{ }{\tt \&}\mbox{ }{\tt |}\mbox{ }{\tt !}\mbox{ }{\tt \sim}\mbox{ }{\tt <}\mbox{ }{\tt >}\mbox{ }{\tt =}\mbox{ }{\tt /}\mbox{ }{\tt \%}\mbox{ }{\tt ?}\mbox{ }{\tt .}\mbox{ }{\tt \$}\mbox{ }{\tt \mathcircumflex}
\]
Note this definition coincides with
symbols in Section 3.1 of the SMT LIB version 2.6 format,
apart from differences in their reserved words.

\subsection{Identifiers}

An identifier $\identifier$
is a syntactic extension of symbols 
that includes symbols that are indexed by integer constants or other symbols.
\[
\begin{array}{rcl}
\identifier & ::= & \begin{array}{ccc}
\symbol & | & \paren{{\tt \_}\mbox{ }\symbol\mbox{ }\kplus{\identifierindex}}
\end{array}\\
\identifierindex & ::= & \begin{array}{ccc}
\intconst & | & \symbol
\end{array}\\
\end{array}
\]
Note this definition coincides with
identifiers in Section 3.1 of the SMT LIB version 2.6 format.

\subsection{Sorts}

We work in a multi-sorted logic where terms 
are associated with sorts $\sortexpr$.
Sorts are constructed via the following syntax.
\begin{alignat*}{1}
 & \begin{array}{rcl}
\sortexpr & ::= & \identifier\mbox{ }|\mbox{ }\paren{\identifier\mbox{ }\kplus{\sortexpr}}\\
\end{array}
\end{alignat*}
The \emph{arity} of the sort is the number of (sort) arguments it takes.
A \emph{parametric} sort is one whose arity is greater than zero.
Theories associate identifiers with sorts and sort constructors
that have an intended semantics.
Sorts may be defined by theories (see examples in Section~\ref{ssec:smt-logic})
or may be user-defined
(see Section~\ref{ssec:defining-sorts}).

\subsection{Terms}

We use terms $\term$ to specify grammars and constraints,
which are constructed by the following syntax.
\begin{alignat*}{1}
 & \begin{array}{rcl}
\term & ::= & \qualidentifier\\
 & | & \literal\\
 & | & \paren{\qualidentifier\mbox{ }\kplus{\term}}\\
 & | & \paren{\existskwd\mbox{ }\paren{\kplus{\sortedvar}}\mbox{ }\term}\\
 & | & \paren{\forallkwd\mbox{ }\paren{\kplus{\sortedvar}}\mbox{ }\term}\\
 & | & \paren{\letkwd\mbox{ }\paren{\kplus{\varbinding}}\mbox{ }\term}\\[2ex]
 \bfterm & ::= & \qualidentifier\\
 & | & \literal\\
 & | & \paren{\qualidentifier\mbox{ }\kplus{\bfterm}}\\[2ex]
 \qualidentifier & ::= & \identifier\mbox{ }|\mbox{ }\paren{\askwd\mbox{ }\identifier\mbox{ }\sortexpr}\\
 \sortedvar & ::= & \paren{\symbol\mbox{ }\sortexpr}\\
 \varbinding & ::= & \paren{\symbol\mbox{ }\term}\\
\end{array}
\end{alignat*}
Above,
we distinguish a subclass of \emph{binder-free} terms $\bfterm$ in the syntax above,
which do not contain bound (local) variables.
Identifiers that comprise terms
may be \emph{qualified} with a type-cast, using the keyword $\askwd$.
Type casts are used for symbols whose type is ambiguous,
such as parametric datatype constructors, e.g. the nil constructor
for a parametric list.

Like sorts, the identifiers that comprise terms
can either be defined by the user or by the theory.
Examples of the latter are given in Section~\ref{ssec:smt-logic}.

\subsection{Commands}

A command $\cmd$ is given by the following syntax.

\[
\begin{array}{rcl}
\cmd 
 & ::= & \paren{\constraintkwd\mbox{ }\term} \\
 & | & \paren{\checksynthkwd} \\
 & | & \paren{\vardeclkwd\mbox{ }\symbol\mbox{ }\sortexpr}\\ 
 & | & \paren{\constraintinvkwd\mbox{ }\symbol\mbox{ }\symbol\mbox{ }\symbol\mbox{ }\symbol} \\
 & | & \paren{\setfeaturekwd\mbox{ }\symbol\mbox{ }\boolconst} \\
 & | & \paren{\synthfunkwd\mbox{ }\symbol\mbox{ }\paren{\kstar{\sortedvar}}\mbox{ }\sortexpr\mbox{ }\koption{\grammardef}}\\
 & | & \paren{\synthinvkwd\mbox{ }\symbol\mbox{ }\paren{\kstar{\sortedvar}}\mbox{ }\koption{\grammardef}} \\
 & | & \smtcmd \\[2ex]
 %\paren{\constdeclkwd\mbox{ }\symbol\mbox{ }\sortexpr} \\
 % ajr: removed
 %\paren{\fundeclkwd\mbox{ }\symbol\mbox{ }\paren{\kstar{\sortexpr}}\mbox{ }\sortexpr} \\
\smtcmd 
 & ::= &\paren{\dtdeclkwd\mbox{ }\symbol\mbox{ }\dtdec} \\
 & | & \paren{\dtsdeclkwd\mbox{ }\paren{\kstarn{\sortdecl}{n+1} }\mbox{ }\paren{\kstarn{\dtdec}{n+1}}} \\
 & | & \paren{\sortdeclkwd\mbox{ }\symbol\mbox{ }\intconst} \\
 & | & \paren{\fundefkwd\mbox{ }\symbol\mbox{ }\paren{\kstar{\sortedvar}}\mbox{ }\sortexpr\mbox{ }\term} \\  %\fundef
 %& | & \paren{\recfundefkwd\mbox{ }\fundef} \\
 %& | & \paren{\recfunsdefkwd\mbox{ }\paren{\kstarn{\fundec}{n+1}}\mbox{ }\paren{\kstarn{\term}{n+1}}} \\
 & | & \paren{\sortdefkwd\mbox{ }\symbol\mbox{ }\sortexpr} \\
 & | & \paren{\setlogickwd\mbox{ }\symbol} \\
 & | & \paren{\setoptkwd\mbox{ }\option} \\[2ex]
 \sortdecl & ::= & \paren{\symbol\mbox{ }\intconst}\\
 %\sortedvar & ::= & \paren{\symbol\mbox{ }\sortexpr}\\  %ajr: used above
 %\fundec & ::= & \paren{\symbol\mbox{ }\paren{\kstar{\sortedvar}}\mbox{ }\sortexpr}\\
 %\fundef & ::= & \symbol\mbox{ }\paren{\kstar{\sortedvar}}\mbox{ }\sortexpr\mbox{ }\term\\
 \dtdec & ::= & \paren{\kplus{\dtconsdec}} \\
 \dtconsdec & ::= & \paren{\symbol\mbox{ }\kstar{\sortedvar}} \\
 \grammardef & ::= & \paren{\kstarn{\sortedvar}{n+1}}\mbox{ }\paren{\kstarn{\ntdef}{n+1}} \\
 \ntdef & ::= & \paren{\symbol\mbox{ }\paren{ \kplus{\gterm} } } \\[2ex]
 \gterm 
 & ::= &  \paren{\constantkwd\mbox{ }\sortexpr}\mbox{ }|\mbox{ }\paren{\varkwd\mbox{ }\sortexpr}\mbox{ }|\mbox{ }\bfterm\\
\end{array}
\]
For conveinence,
we distinguish two kinds of commands above.
The commands listed under $\cmd$ are specific to the SyGuS version 2.0 standard.
The remaining commands listed under $\smtcmd$
are borrowed from the SMT LIB version 2.6 standard.
Details on the semantics of these commands are given in Section~\ref{sec:semantics}.

%\paragraph{Reserved Words}
%All strings mentioned in Section~\ref{ssec:literals}

\section{Semantics of Commands}
\label{sec:semantics}

A SyGuS input file is a sequence of commands,
which at a high level
are used for defining a (single) synthesis conjecture
and invoking a solver for this conjecture.
This conjecture is of the form:
\begin{alignat*}{1}
 & \exists f_1,\ldots,f_n.\, \forall x_1,\ldots,x_m.\,(\varphi_1 \wedge \ldots \wedge \varphi_p )[f_1,\ldots,f_n,x_1,\ldots,x_m]
\end{alignat*}
In this section, we define how this conjecture is
is established via SyGuS commands.
%At a high level, 
%functions to synthesize $f_1, \ldots, f_n$ 
Given a sequence of commands, the current state consists of the following
information:
\begin{itemize}
\item A set
$f_1, \ldots, f_n$, which we refer to
as the current set of \emph{functions to synthesize},
\item A set 
$x_1, \ldots, x_m$ of variables, 
which we refer to the current set of \emph{universal variables}, 
\item A set of formulas
$\varphi_1, \ldots, \varphi_p$,
which we refer to as the current set of \emph{constraints},
\item A \emph{signature}
denoting the set of defined symbols in the current scope.
A signature is 
a mapping from symbols to expressions (either sorts or terms).
Each of these symbols may have a predefined semantics
either given by the theory,
or defined by the user (e.g. symbols that are defined as macros
fit the latter category).
\item A \emph{logic} denoting the
language of terms that may appear in constraints and grammars.
\end{itemize}

In the initial state of a SyGuS input,
the sets of functions-to-synthesize, 
universal variables, constraints, and the signature is empty,
and the logic is unspecified,
that is, 
no restrictions are placed on the set of terms that may appear 
in constraints and grammars.

In the following, we first describe restrictions
on the order in which commands can be specified in SyGuS inputs.
We then describe how each command $\cmd$ updates
the state of the sets above and the current signature.

\subsection{Command Ordering}

A SyGuS input is not well-formed
if it specifies a list of commands that do not meet
the restrictions given in this section regarding their order.

\rem{(Set logic | set extension | set option )* ( everything else )* ( check-synth )? }
%\rem{Order of commands}

\subsection{Setting the Logic}
\label{ssec:set-logic}

The logic of SyGuS consists of two parts,
a \emph{base logic} and a \emph{feature set},
which are described in detail in Section~\ref{sec:sygus-logic}.

\begin{itemize}
\item $\paren{\setlogickwd\mbox{ }S}$

This sets the base logic of the background logic
to the one that $S$ refers to,
which can be a standard one defined in SMT-LIB~\cite{}
or may be solver-specific.
For examples of common SMT logics, 
see Section~\ref{ssec:smt-logic}.
If $S$ is a standard SMT-LIB logic,
then this command adds the set of sort and function symbols
from that logic to the current signature.
In this case,
the logic must contain quantifiers or this command is not well-formed,
that is, logics with the prefix ``${\tt QF\_}$'' are not allowed.

In SyGuS, notice that the logic gives
restrictions on both the kinds of terms that
may appear in constraints, and may appear in grammars.
We give details on the default settings of the logic for SyGuS inputs
in Section~\ref{sec:sygus-logic}.

\item $\paren{\setfeaturekwd\mbox{ }S\mbox{ }\boolconst}$

This enables or disables the feature specified by $S$
in the feature set of the background logic.
This may correspond to affect the set of commands 
allowable in an input,
or further refinements to the restrictions placed
on the set of allowable constraints and grammars.
All features standardized by this document 
are given in Section~\ref{sec:sygus-logic}.

\end{itemize}

\subsection{Declaring Universal Variables}

\begin{itemize}
\item $\paren{\vardeclkwd\mbox{ }S\mbox{ }\sigma}$

This command adds $S$ to the current set of universal variables
and adds the symbol $S$ of sort $\sigma$ to the current signature.
This command should be rejected if $S$ is already
a symbol in the current signature.

%\item $\paren{\constdeclkwd\mbox{ }\symbol\mbox{ }\sortexpr}$

%\item $\paren{\fundeclkwd\mbox{ }S\mbox{ }\paren{\sigma_1 \ldots \sigma_n}\mbox{ }\sigma}$

%This command adds $S$ to the current set of universal variables.
%The sort of $S$ is $\sigma$ if $n=0$ or
%$\sigma_1 \times \ldots \times \sigma_n \rightarrow \sigma$ if $n>0$.
%This command should be rejected if $S$ is already
%a symbol in the current signature.

%\rem{This is higher-order quantification.
%TODO: rename this ``declare-var-fun''? Parallel with declare-const -> declare-var.
%Maybe remove this command.
%}

\end{itemize}

\subsection{Declaring Functions-to-Synthesize}
\label{ssec:dec-synth-fun}

\begin{itemize}
\item $\paren{\synthfunkwd\mbox{ }S\mbox{ }
\paren{\paren{x_1\mbox{ }\sigma_1} \ldots \paren{x_n\mbox{ }\sigma_n} }\mbox{ }\sigma\mbox{ }
\koption{G}}$

This command adds $S$ to the current set of functions to synthesize,
and adds the symbol $S$ of 
sort $\sigma_1 \times \ldots \times \sigma_n \rightarrow \sigma$
to the current signature.
This command should be rejected if $S$ is already
a symbol in the current signature.
We describe restrictions and well-formedness requirements for this command
in the following.

If provided, the syntax for the grammar $G$
consists of two parts: 
a \emph{predeclaration}
$\paren{\paren{y_1\mbox{ }\tau_1} \ldots \paren{y_n\mbox{ }\tau_n}}$,
followed by a \emph{grouped rule listing}
$\paren{\paren{y_1\mbox{ }\paren{g_{11} \ldots g_{1m_1} } } \ldots \paren{y_n\mbox{ }\paren{g_{n1} \ldots g_{nm_n} } }}$
where $y_1, \ldots, y_n$ are the \emph{non-terminal} symbols of the grammar.
Note that the same variable symbols
$y_1, \ldots, y_n$ appear both in the predeclaration and as heads
of each of the rules.
If this is not the case, then this command is not well-formed.
For all $i,j$, recall that grammar term $g_{ij}$ is either a term, or a class of terms
denoted by $\paren{\constantkwd\mbox{ }\sigma_c}$ 
and $\paren{\varkwd\mbox{ }\sigma_v}$ denoting respectively
the set of constants whose sort is $\sigma_c$,
and the set of all variables from $x_1, \ldots, x_n$ whose sort is $\sigma_v$.
If $g_{ij}$ is an ordinary term, then its free variables may contain $y_1, \ldots, y_n$,
as well as $S$ itself, although the latter is disallowed in the default logic (see Section~\ref{sec:sygus-logic}).

This command is not well-formed if
$\tau_1$ (the type of the start symbol) is not $\sigma$.
%\rem{TODO: should we make the first symbol of list the start symbol, instead of distinguished Start keyword?},
It is also not well-formed if
$G$ generates a term $t$ from $y_i$
that does not have type $\tau_i$ for some $i$.
Details on grammars and the terms
they generate is discussed in more detail in Section~\ref{ssec:sat-syntactic}.

The grammar $G$ must also be one that is allowable by $\mathcal{L}$.
For more details on the restrictions imposed on grammars by the logic,
see Section~\ref{sec:sygus-logic}.
If $G$ does not meet the restrictions
of the background logic, it should be rejected.

If $G$ is not provided,
then this command is syntax
sugar for
$\paren{\synthfunkwd\mbox{ }S\mbox{ }
\paren{\paren{x_1\mbox{ }\sigma_1} \ldots \paren{x_n\mbox{ }\sigma_n} }\mbox{ }\sigma\mbox{ }
G_{\mathcal{L},\vec{x},\sigma}}$
where $G_{\mathcal{L},\vec{x},\sigma}$ is a grammar that
generates all well-sorted terms of sort $\sigma$ 
that belong to the language of the current background logic $\mathcal{L}$,
and whose free variables are in $\vec{x} = (x_1, \ldots, x_n)$.
%We call $G_{\mathcal{L},\sigma,\vec{x}}$ the \emph{default}
%grammar for $(T,\sigma,\vec{x})$.

\item $\paren{\synthinvkwd\mbox{ }S\mbox{ }\paren{\paren{x_1\mbox{ }\sigma_1} \ldots \paren{x_n\mbox{ }\sigma_n} }\mbox{ }\koption{G}}$

This is syntax sugar for
$\paren{\synthfunkwd\mbox{ }S\mbox{ }\paren{\paren{x_1\mbox{ }\sigma_1} \ldots \paren{x_n\mbox{ }\sigma_n} }\mbox{ }\boolkwd\mbox{ }\koption{G}}$.
\end{itemize}

\subsection{Defining Sorts}
\label{ssec:defining-sorts}

\begin{itemize}
\item $\paren{\dtdeclkwd\mbox{ }S\mbox{ }D}$

This is syntax sugar for
$\paren{\dtsdeclkwd\mbox{ }\paren{\paren{S\mbox{ }0} }\mbox{ }\paren{D}}$.

\item $\paren{\dtsdeclkwd\mbox{ }\paren{\paren{S_1\mbox{ }a_1}\ldots\paren{S_n\mbox{ }a_n} }\mbox{ }\paren{D_1 \ldots D_n}}$

This command adds symbols corresponding to the datatype
definitions $D_1, \ldots, D_n$ for $S_1, \ldots, S_n$ 
to the current signature.
For each $i = 1, \ldots, n$, integer constant $a_i$ denotes the arity of 
datatype $S_i$.
Recall the syntax of 
$D_i$ is a \emph{constructor listing} of the form
\[
\paren{
\paren{c_1\mbox{ }\paren{s_{11}\mbox{ }\sigma_{11}}\mbox{ }\ldots\mbox{ }\paren{s_{1m_1}\mbox{ }\sigma_{1m_1}} }
\mbox{ }\ldots\mbox{ }
\paren{c_k\mbox{ }\paren{s_{k1}\mbox{ }\sigma_{k1}}\mbox{ }\ldots\mbox{ }\paren{s_{km_k}\mbox{ }\sigma_{km_k}} }
}
\]
For each $i$, the following symbols are added to the signature:
\begin{enumerate}
\item 
Symbol $S_i$ is added to the current signature,
defined it as a datatype sort whose definition is given by $D_i$,
\item 
Symbols $c_1, \ldots, c_k$ are added to the signature,
where
for each $j = 1, \ldots, k$, symbol $c_j$
is defined as a \emph{constructor}
of sort $\sigma_{j1} \times \ldots \times \sigma_{jm_j} \rightarrow D_i$,
\item 
For each $j = 1, \ldots, k$, $\ell = 1, \ldots m_j$, symbol $s_{j\ell}$
is added to the signature,
defined as a \emph{selector}
of sort $D_i \rightarrow \sigma_{j\ell}$.
%For each $j = 1, \ldots, k$, 
\end{enumerate}

%\rem{Explain symbols?}
This command should be rejected if any of the above symbols this command
adds to the signature are already
a symbol in the current signature.
We provide examples of datatype definitions in Section~\ref{sec:examples}.
For full details on well-formed datatype declarations,
refer to Section 4.2.3 of the SMT LIB version 2.6~\cite{}.

\item $\paren{\sortdeclkwd\mbox{ }S\mbox{ }n}$

This command adds the symbol $S$ to the current signature
and associates it with an uninterpreted sort of arity $n$.
This command should be rejected if $S$ is already
a symbol in the current signature.

\end{itemize}

\subsection{Defining Macros}

\begin{itemize}
\item $\paren{\fundefkwd\mbox{ }S\mbox{ }\paren{\paren{x_1\mbox{ }\sigma_1} \ldots \paren{x_n\mbox{ }\sigma_n}}\mbox{ }\sigma\mbox{ }t}$

This adds to the current signature
the symbol $S$ of sort $\sigma$
if $n=0$ or $\sigma_1 \times \ldots \sigma_n \rightarrow \sigma$ if $n>0$.
The variables $x_1, \ldots, x_n$ may occur freely in $t$.
It defines $S$ as a term whose semantics are given by the function
$\lambda x_1, \ldots, x_n.\, t$.
Notice that $t$ may not contain any free occurrences of $S$,
that is, the definition above is not recursive.
This command is not well-formed if $t$ is not a well-sorted
term of sort $\sigma$.
This command should be rejected if $S$ is already
a symbol in the current signature.

\item $\paren{\sortdefkwd\mbox{ }S\mbox{ }\paren{u_1 \ldots u_n}\mbox{ }\sigma}$

This adds the symbol $S$ to the current signature.
It defines $S$ as the sort $\sigma$.
The sort variables $u_1, \ldots, u_n$
may occur free in $\sigma$,
while $S$ may not occur free in $\sigma$.
This command is not well-formed if $\sigma$
is not a well-formed sort.
This command should be rejected if $S$ is already
a symbol in the current signature.

\end{itemize}

\subsection{Asserting Synthesis Constraints}

\begin{itemize}
\item $\paren{\constraintkwd\mbox{ }t}$

This adds $t$ to the set of constraints.
This command is well formed if $t$ is a well-sorted formula,
that is, a term of sort $\sbool$.
% constraint
%We further require that $t$ is in the logic of
Furthermore,
the term $t$ should be allowable
based on the restrictions of the current logic,
see Section~\ref{sec:sygus-logic} for more details.

\item $\paren{\constraintinvkwd\mbox{ }S\mbox{ }S_{pre}\mbox{ }S_{trans}\mbox{ }S_{post}}$

%This command is syntax-sugar for
%declaring a set of variables 
%and asserting a constraint corresponding to the invariant synthesis problem.
A constraint of this form is well-formed if:
\begin{enumerate}
\item
$S$ is function-to-synthesize
of sort $\sigma_1 \times \ldots \times \sigma_n \rightarrow \sbool$,
\item
$S_{pre}$ is a defined symbol 
whose definition is of the form $\lambda x_1, \ldots, x_n.\, \varphi_{pre}$,
\item
$S_{trans}$ is a defined symbol 
whose definition is of the form $\lambda x_1, \ldots, x_n, y_1, \ldots, y_n.\, \varphi_{trans}$, and
\item
$S_{post}$ is a defined symbol 
whose definition is of the form $\lambda x_1, \ldots, x_n.\, \varphi_{post}$.
\end{enumerate}
where $(x_1, \ldots, x_n)$ and $(y_1, \ldots, y_n)$
are tuples of variables of sort $(\sigma_1, \ldots, \sigma_n)$ and
$\varphi_{pre}$, $\varphi_{trans}$ and $\varphi_{post}$ are formulas.
Given that this command is well-formed, given the above definitions,
this command is syntax sugar for:
\[
\begin{array}{l}
\paren{\vardeclkwd\mbox{ }v_1\mbox{ }\sigma_1}\\
\paren{\vardeclkwd\mbox{ }v'_1\mbox{ }\sigma_1}\\
\ldots\\
\paren{\vardeclkwd\mbox{ }v_n\mbox{ }\sigma_n}\\
\paren{\vardeclkwd\mbox{ }v'_n\mbox{ }\sigma_n}\\
\paren{\constraintkwd\mbox{ }\paren{{\tt =>}\mbox{ }
\paren{S_{pre}\mbox{ }v_1\mbox{ }\ldots\mbox{ }v_n}\mbox{ }
\paren{S\mbox{ }v_1\mbox{ }\ldots\mbox{ }v_n}}}\\
\paren{\constraintkwd\mbox{ }\paren{{\tt =>}\mbox{ }
\paren{{\tt and}\mbox{ }\paren{S\mbox{ }v_1\mbox{ }\ldots\mbox{ }v_n}\mbox{ }
\paren{S_{trans}\mbox{ }v_1\mbox{ }\ldots\mbox{ }v_n\mbox{ }v'_1\mbox{ }\ldots\mbox{ }v'_n}}\mbox{ }
\paren{S\mbox{ }v'_1\mbox{ }\ldots\mbox{ }v'_n}}}\\
\paren{\constraintkwd\mbox{ }\paren{{\tt =>}\mbox{ }
\paren{S\mbox{ }v_1\mbox{ }\ldots\mbox{ }v_n}\mbox{ }
\paren{S_{post}\mbox{ }v_1\mbox{ }\ldots\mbox{ }v_n}}}\\
\end{array}
\]
where $v_1, v'_1, \ldots, v_n, v'_n$ are fresh symbols.
The above set of assertions has the form of an invariant synthesis problem for $S$,
where $S_{pre}$ denotes a pre-condition,
$S_{post}$ denotes a post-condition
and $S_{trans}$ denotes a transition relation.

\end{itemize}

\subsection{Initiating Synthesis Solver}

\begin{itemize}
\item $\paren{\checksynthkwd}$

This asks the synthesis solver to find a solution for the synthesis conjecture
corresponding to the current set of functions-to-synthesize,
universal variables and constraints.
The expected output from the synthesis solver is covered in Section~\ref{sec:output}.
\end{itemize}

\subsection{Setting Solver Options}

\begin{itemize}
\item $\paren{\setoptkwd\mbox{ }O}$

This enables the solver-specific options specified by $O$.
We do not give concrete examples of such options in this document. %\rem{Any needed?}
It is recommended that synthesis solvers
ignore unrecognized options, 
and choose reasonable defaults when the
options are left unspecified.
\end{itemize}

\section{Synthesis Solver Output}
\label{sec:output}

This section covers the expected output from a synthesis solver,
which currently is limited to responses to $\checksynthkwd$ 
only.

\begin{itemize}
\item
A well-formed response to $\checksynthkwd$ is one of the following:

\begin{enumerate}
\item
A list of $\fundefkwd$ commands of the form:
\[
\begin{array}{l}
\paren{\fundefkwd\mbox{ }f_{o1}\mbox{ }X_{o1}\mbox{ }\sigma_{o1}\mbox{ }t_{o1}}\\
\ldots\\
\paren{\fundefkwd\mbox{ }f_{on}\mbox{ }X_{on}\mbox{ }\sigma_{on}\mbox{ }t_{on}}
\end{array}
\]
where functions $\{ f_{o1}, \ldots, f_{on} \}$
are the functions-to-synthesize $\{ f_1,\ldots,f_n \}$ (in any order), 
$X_{o1}, \ldots, X_{on}$ are sorted variable lists,
$\sigma_{o1}, \ldots, \sigma_{on}$ are types,
and $t_{o1}, \ldots, t_{on}$ are terms.
To be a well-formed output, 
for each $j=1, \ldots, n$, it must be the case that
$t_{oj}$ is a term of $\sigma_{oj}$,
and $X_{oj}$ is identical to the sorted variable list
used when introducing the function-to-synthesize $f_{oj}$.

\item
The output ${\tt (fail)}$.

\end{enumerate}

A response to $\checksynthkwd$ is a correct solution if
it satisfies both the semantic and syntactic restrictions given by
the current state.
We describe this in more detail in Section~\ref{sec:logical-semantics}.

\end{itemize}

%\section{

\section{SyGuS Background Logics}
\label{sec:sygus-logic}

In this section,
we describe how the background logic
restricts the constraints and grammars that are allowable as inputs.
%As mentioned in Section~\ref{ssec:set-logic},
%a SyGuS input may be given a background logic
%specified by a symbol $S$,
%where $S$ may be a standard SMT-LIB logic
%or may be solver-specific.
%We refer to $S$ as the \emph{base logic}
%of a SyGuS input.
A SyGuS background logic consists of two parts:
\begin{enumerate}
\item A \emph{base logic}, which can be set via a $\setlogickwd$ command.
This corresponds to a SMT-LIB standard logic or may be solver specific.

\item A \emph{feature set}, 
which augments the background
logic with additional restrictions or language constructs.
Features are not expressible in a base logic
and are generally parametric to the base logic.
\end{enumerate}
Both components of the logic affect the kinds of terms that may appear
in constraints and in grammars.
%We give more detail on what consistutes an extended logic in the following.

The following
subsection gives an overview of the kinds of base logics that
are defined in SMT-LIB and the feature sets
standardized by this document.
The next two subsections 
summarize how these
restrict the set of terms that may appear 
in constraints and grammars.
%We give three such examples in Section~\ref{ssec:logicr-grammars}
%below.
%\rem{Standard extension to base logic + extensions?}
%This section pertains only to 

\subsection{Base Logics}
\label{ssec:smt-logic}

%This section covers only base logics defined in the SMT-LIB standard.
SMT-LIB provides a catalog of standard logics,
available at \url{www.smt-lib.org}.
Further details on the formal definition of 
theory and logic declarations can be found in Sections 3.7 and 3.8
of the SMT-LIB version 2.6 standard~\cite{}.
At a high level, a \emph{logic}
enables a set of theories.
If a logic enables a theory, then its symbols are added
to the current signature when a $\setlogickwd$ command is issued.
We briefly review some of the important theories and logics in the following.

The theory of integers is enabled 
in logics like linear integer arithmetic ${\tt LIA}$
or non-linear integer arithmetic ${\tt NIA}$.
The signature of this theory includes the integer sort ${\tt Int}$ and
typical function symbols of arithmetic, including
addition ${\tt +}$
and multiplication ${\tt *}$.
Unary negation and subtraction are specified by ${\tt -}$.
Constants of the theory are integer constants whose syntax
is given by $\intconst$ from Section~\ref{ssec:literals}.
Analogously, 
the theory of reals is enabled
in logics like linear real arithmetic ${\tt LRA}$
or non-linear real arithmetic ${\tt NRA}$.
It signature includes the real sort ${\tt Real}$.
Some of the function symbols of arithmetic are syntacticaly identical to
those from the theory of integers, including
${\tt +}$, ${\tt -}$ and ${\tt *}$.
The signature of the theory of reals additionally includes
real division ${\tt /}$.
Constants in this theory can either be specified as
decimals using the syntax $\realconst$
or as rational terms of the form ${\tt (/\ m\ n)}$.


The theory of fixed-width bit-vectors
is included in logics specified 
by symbols that include the substring ${\tt BV}$.
The signature of this theory includes a family of 
indexed sorts ${\tt (\_\ BitVec\ n)}$
denoting bit-vectors of width $n$.
The functions in this signature include various operations on bit-vectors,
including bit-wise, arithmetic, and shifting operations.

\rem{strings? all?}

More details on the concrete signatures introduced by
standard SMT-LIB logics can be found in the reference
grammars in Appendix~\ref{apx:ref-grammars}.

The default logic is the empty logic,
that is, no restrictions are placed on the set of terms that may 
appear, nor are any symbols predefined.

\subsection{Feature Sets}
\label{ssec:feature-sets}

For the purposes of this document, 
a feature set can be seen as a set of values, called \emph{features},
taken from the following enumeration.
\[
\equantifierskwd\mbox{ }|\mbox{ }\efwddeclskwd\mbox{ }|\mbox{ }\erecursionkwd
\]
\rem{TODO: move to syntax section?}
We say that a feature $F$ is enabled in the background logic
if its feature set contains $F$.
The above features indicate (respectively) whether
arbitrary quantification is allowed in constraints,
grammars of $\synthfunkwd$ may refer to previously declared synthesis functions,
and whether grammars can generate terms that correspond to recursive definitions.

The default extended logic is the empty set and can be modified
by the $\setfeaturekwd$ command.

\subsection{SyGuS Logic Restrictions on Constraints}
\label{ssec:logicr-constraints}

%For the former,
%we follow the same conventions as SMT-LIB
%if $S$ is a standard SMT-LIB logic.
%For the latter,
%we give details on the default settings of the logic
%with respect to restrictions on terms that may appear in grammars
%in Section~\ref{sec:sygus-logic}.

%This section defines restrictions on the terms $t$
%that can appear in commands of the form $\paren{\constraintkwd\mbox{ }t}$.
Assuming our base logic is a standard SMT-LIB background logic,
a term $t$ is allowable in this context
generally if it belongs to that logic, as prescribed by the SMT-LIB standard.
Exceptions to this rule are described here.

By convention as stated in Section~\ref{ssec:set-logic}, 
the base logic specified
by a $\setlogickwd$ in a SyGuS file must allow quantifiers.
This is due to the fact that
the synthesis conjecture implicit in the state
contains universal quantification on the universal variables of our state.
However,
constraints \emph{cannot} themselves contain 
formulas with quantifiers
unless the feature $\equantifierskwd$ is enabled in the background logic.

%\rem{Reference extension?}

\subsection{SyGuS Logic Restrictions on Grammars}
\label{ssec:logicr-grammars}

%This section defines restrictions on the terms
%that may appear in grammars.
%The base logic affects the kinds of grammars that
%are allowable in SyGuS inputs.
A grammar $G$ is not allowable by the base logic if
it generates some term $t$ that does not belong to that logic.
%For instance, 
%this restriction disallows grammars
%in the logic of \emph{linear} integer arithmetic
%for which it is possible to generate a term involving
%the multiplication of two non-constant integer terms.
It may be the case that a grammar $G$
contains a rule whose conclusion does not itself meet the restrictions 
of the background logic.
For example, consider the logic of \emph{linear} integer arithmetic
and a grammar $G$ containing a non-terminal symbol $y_c$ of integer type
such that $G$ generates only constants from $y_c$.
Grammar $G$ may be allowed in the logic of linear integer arithmetic
even if it has a rule whose conclusion is $y_c * t$, 
noting that no non-linear terms can be generated from this rule,
provided that no non-linear terms can be generated from $t$.
%then the term $y_c * y$ may be the conclusion
%grammar terms $g_{ij}$ may involve
%the multiplication of two non-constant integer terms $t_1$ and $t_2$
%e.g. in the case there one of $t_i$ is a non-terminal symbol $y$
%for which $G$ generates only constant integer terms from $y$.

%The default logic assumed by the SyGuS standard assumes
%additional restrictions on the terms that are generated by grammars.
The feature set component of the background logic 
may impose additional restrictions
on the terms that are generated by grammars.
Note the following definition.
The \emph{expanded form} of a term $t$
is the (unique) term obtained
by replacing all functions $f$ in $t$ that are defined as macros
%in the current signature
with their corresponding definition until a fixed point is reached.
A grammar for function-to-synthesize $f$ is not allowed
if it contains a rule whose conclusion is a term $t$
whose expanded form contains applications of functions-to-synthesize
unless the feature $\efwddeclskwd$ is enabled;
it is not allowed if $t$ 
contains $f$ itself unless the feature $\erecursionkwd$ is enabled.
%Similarly, a grammar for function-to-synthesize $f$
%is not allowed by the extended logic
%if it contains a rule whose conclusion is a term $t$
%\rem{TODO}
%A grammar that generates a term whose expanded
%form contains a function-to-synthesize
%is not allowed unless $\efwddeclskwd$ is enabled in the extended logic.
%A grammar for a function-to-synthesize $f$
%that generates a term that contains $f$
%is not allowed unless $\

%\rem{forward declarations, recursion}
%Note the following (recursive) definition.
%A symbol $S$ \emph{depends on functions-to-synthesize}
%if $S$ is a function-to-synthesize
%or if it is a defined symbol whose definition $\lambda \vec{x}.\,t$
%is such that $t$ contains a symbol that depends on
%functions-to-synthesize.
%We say that $G$ has \emph{forward declarations}
%if 
%We say that $G$ is \emph{recursive} 
%if 
%\rem{TODO}



\rem{Special logics: single invariant synthesis?}

\section{Formal Semantics}
\label{sec:logical-semantics}

Here we give the formal semantics
for what consistites a correct solution for synthesis conjecture.

\subsection{Satisfying Syntactic Specifications}
\label{ssec:sat-syntactic}

In this section,
we formalize the notion of satisfying the \emph{syntactic specification}
of the synthesis conjecture.

As described in Section~\ref{ssec:dec-synth-fun},
a grammar $G$ is specified as
a \emph{grouped rule listing} of the form
\[
\paren{\paren{y_1\mbox{ }\paren{g_{11} \ldots g_{1m_1} } } \ldots 
\paren{y_n\mbox{ }\paren{g_{n1} \ldots g_{nm_n} } }}
\]
where $y_1, \ldots, y_n$ are variables
and $g_{11} \ldots g_{1m_1}, \ldots, g_{n1} \ldots g_{nm_n}$
are grammar terms.
We associate each grammar with a sorted variable list $X$,
namely the argument list of the function-to-synthesize.
We refer to $y_1$ as the \emph{start symbol} of $G$.

We interpret $G$ as a (possibly infinite) set of rules 
of the form $y \mapsto t$ where $t$ is an (ordinary) term
based on the following definition.
For each $y_i, g_{ik}$ in the grouped rule list,
if $g_{ik}$ is $\paren{\constantkwd\mbox{ }\sigma_c}$,
then $G$ contains the rule $y_i \mapsto c$ for all constants of sort $\sigma_c$.
If $g_{ik}$ is $\paren{\varkwd\mbox{ }\sigma_v}$,
then $G$ contains the rule $y_i \mapsto x$
for all variables $x \in X$ of sort $\sigma_v$.
Otherwise, if $g_{ik}$ is an ordinary term,
then $G$ contains the rule $y_i \mapsto g_{ik}$.

We say that $G$ \emph{generates} term $r$ from $s$
if it is possible to construct a sequence of terms
$s_1, \ldots, s_n$
with $s_1 = s$ and $s_n = r$
where for each $1 \leq i \leq n$, term $s_i$ is obtained from $s_{i-1}$ by
replacing an occurrence of some $y$ by $t$
where $y \mapsto t$ is a rule in $G$.

Let $f$ be a function to synthesize,
which recall is a grammar $G$
that are both associated with a common variable list $X$.
A term $\lambda X.\, t$
satisfies the syntactic specification for $f$
if $G$ generates $t$ starting from $y_1$,
where $y_1$ is the start symbol of $G$.
%\rem{More}
%whose variable lists are $X_1, \ldots, X_n$
%and whose grammars are $G_1, \ldots, G_n$.

%\begin{def}
A tuple of functions 
$( \lambda X_1.\, t_1, \ldots, \lambda X_n.\, t_n )$
satisfies the syntactic restrictions
for functions-to-synthesize $( f_1, \ldots, f_n )$
in conjecture $\exists f_1, \ldots, f_n \varphi$
if $\lambda X_i.\, t_i$
satisfies the syntactic specification for $f_i$
for each $i=1,\ldots,n$.
%\end{def}

\subsection{Satisfying Semantic Specifications}
\label{ssec:sat-semantic}

In this section,
we formalize the notion of satisfying the \emph{semantic specification}
of the synthesis conjecture
for background theories from the SMT-LIB standard.
%\rem{Reference smt semantics}

%\begin{def}
A tuple of functions 
$( \lambda X_1.\, t_1, \ldots, \lambda X_n.\, t_n)$
satisfies the semantic restrictions
for functions-to-synthesize $( f_1, \ldots, f_n )$
in conjecture $\exists f_1, \ldots, f_n \varphi$ in background theory $T$
if $\psi$ is $T$-valid formula,
where $\psi$ is the formula
$\varphi \{ f_1 \mapsto \lambda X_1.\, t_1, \ldots, f_n \mapsto \lambda X_n.\, t_n \}$
after beta-reduction.
%which notice contains no 
%\end{def}

The notion of satisfying semantic restrictions for background theories
that not specified in the SMT-LIB standard are not covered by this document.


\section{Examples}
\label{sec:examples}

\rem{Simple LIA example}

\rem{extract/concat: multiple types}

\rem{Datatypes}

\rem{Multiple check-synth}

\bibliographystyle{plain}
\bibliography{references}

\begin{appendix}

\section{Reference Grammars}
\label{apx:ref-grammars}

In this section, for conveinence, we provide the concrete
grammar corresponding to the most premissive grammars for generating
terms that belong to SMT-LIB logics of interest.
Note this is not intended to be a complete list.
Each of these grammars are derived
based on the definition of logics and theories
described in the theory and logic declaration documents
available at \url{www.smt-lib.org}.

For each grammar, we omit 
the predicate symbols that are shared by all logics
according to the SMT-LIB standard. This includes
the equality predicate
${\tt =}$ for each sort,
%\item An if-then-else term ${\tt ite}$
and Boolean connectives such as
${\tt and}$, ${\tt or}$, ${\tt =>}$ and ${\tt xor}$.
%All logics include an if-then-else term ${\tt ite}$,
%which are included 
We provide the grammar
for a single function over one of the sorts in the logic,
and assume it has one variable in its argument list
for each non-Boolean sort that the grammar contains.

\subsection{Integer Arithmetic}

The following grammar for $f$
generates exactly the integer-typed terms in the logic of
linear integer arithmetic (LIA) with one free integer variable ${\tt x}$.
\begin{lstlisting}[basicstyle={\ttfamily}]
(set-logic LIA)
(synth-fun f ((x Int)) Int
   ((y_term Int) (y_cons Int) (y_pred Bool))
   ((y_term Int  (y_cons
                  (Variable Int)
                  (- y_term)
                  (+ y_term y_term)
                  (- y_term y_term)
                  (* y_cons y_term)
                  (* y_term y_cons)
                  (div y_term y_cons)
                  (mod y_term y_cons)
                  (abs y_term)
                  (ite y_pred y_term y_term)))
    (y_cons Int  ((Constant Int)))
    (y_pred Bool ((> y_term y_term)
                  (>= y_term y_term)
                  (< y_term y_term
                  (<= y_term y_term))))))
\end{lstlisting}
Above, ${\tt div}$
denotes integer division,
${\tt mod}$ denotes integer modulus and ${\tt abs}$ denotes
the absolute value function.
The following grammar for $g$ generates exactly the integer-typed terms in the logic of
non-linear integer arithmetic (NIA).
\begin{lstlisting}[basicstyle={\ttfamily}]
(set-logic NIA)
(synth-fun g ((x Int)) Int
   ((y_term Int) (y_pred Bool))
   ((y_term Int  ((Constant Real) 
                  (Variable Int)
                  (- y_term)
                  (+ y_term y_term)
                  (- y_term y_term)
                  (* y_term y_term)
                  (div y_term y_term)
                  (mod y_term y_term)
                  (abs y_term)
                  (ite y_pred y_term y_term)))
    (y_pred Bool (...
                  (> y_term y_term)
                  (>= y_term y_term)
                  (< y_term y_term
                  (<= y_term y_term))))))
\end{lstlisting}

\subsection{Real Arithmetic}

The following grammar for $f$
generates exactly the real-typed terms in the logic of
linear real arithmetic (LRA).
\begin{lstlisting}[basicstyle={\ttfamily}]
(set-logic LRA)
(synth-fun f ((x Real)) Real
   ((y_term Real) (y_cons Real) (y_pred Bool))
   ((y_term Real (y_cons 
                  (Variable Real) 
                  (- y_term)
                  (+ y_term y_term)
                  (- y_term y_term)
                  (* y_cons y_term)
                  (* y_term y_cons)
                  (ite y_pred y_term y_term)))
    (y_cons Real ((Constant Real)))
    (y_pred Bool (...
                  (> y_term y_term)
                  (>= y_term y_term)
                  (< y_term y_term
                  (<= y_term y_term))))))
\end{lstlisting}
Notice that real constants can either be specified 
as decimal values using the syntax $\realconst$
or as rationals, e.g. the division of two integer constants
${\tt (/\ m\ n)}$.

The grammar for $g$
generates exactly the real-typed terms in the logic of
non-linear real arithmetic (NRA).
\begin{lstlisting}[basicstyle={\ttfamily}]
(set-logic NRA)
(synth-fun g ((x Real)) Real
   ((y_term Real) (y_cons Real) (y_pred Bool))
   ((y_term Real ((Constant Real) 
                  (Variable Real)
                  (- y_term)
                  (+ y_term y_term)
                  (- y_term y_term)
                  (* y_term y_term)
                  (/ y_term y_term)
                  (ite y_pred y_term y_term)))
    (y_pred Bool (...
                  (= y_term y_term)
                  (> y_term y_term)
                  (>= y_term y_term)
                  (< y_term y_term
                  (<= y_term y_term))))))
\end{lstlisting}

\subsection{Fixed-Width Bit-Vectors}

The signature of bit-vectors includes an indexed sort
${\tt Bitvec}$, which is indexed by an integer constant that denotes
its bit-width. We show a grammar below for one particular
choice of this bitwidth, $32$.
We omit indexed bit-vector operators
such as extraction function ${\tt (\_\ extract\ n\ m)}$, 
the concatenation function ${\tt concat}$,
since these operators are polymorphic.
For brevity,
we also omit the \emph{extended} operators of this theory
as denoted by SMT-LIB,
which includes functions like
bit-vector subtraction ${\tt bvsub}$,
xor ${\tt bvxor}$,
signed division ${\tt bvsdiv}$,
and predicates like
unsigned-greater-than-or-equal ${\tt bvuge}$
and signed-less-than ${\tt bvslt}$.
These extended operators can be seen as syntax sugar for the
operators in the grammar below.
Examples of some of the omitted operators are given in Section~\ref{sec:examples}.

% ajr: full grammar 
\begin{comment}
\begin{lstlisting}[basicstyle={\ttfamily}]
(set-logic BV)
(synth-fun f ((x (_ BitVec 32))) (_ BitVec 32)
   ((y_term (_ BitVec 32)) (y_pred Bool))
   ((y_term (_ BitVec 32) ((Constant (_ BitVec 32))
                           (Variable (_ BitVec 32))
                           (bvnot y_term)
                           (bvand y_term y_term)
                           (bvor y_term y_term)
                           (bvneg y_term)
                           (bvadd y_term y_term)
                           (bvmul y_term y_term)
                           (bvudiv y_term y_term)
                           (bvurem y_term y_term)
                           (bvshl y_term y_term)
                           (bvlshr y_term y_term)
                           (bvnand y_term y_term)
                           (bvnor y_term y_term)
                           (bvxor y_term y_term)
                           (bvxnor y_term y_term)
                           (bvsub y_term y_term)
                           (bvsdiv y_term y_term)
                           (bvsrem y_term y_term)
                           (bvsmod y_term y_term)
                           (bvashr y_term y_term)
                           (ite y_pred y_term y_term)))
    (y_pred Bool (...
                  (bvult y_pred y_pred)
                  (bvule y_pred y_pred)
                  (bvugt y_pred y_pred)
                  (bvuge y_pred y_pred)
                  (bvslt y_pred y_pred)
                  (bvsle y_pred y_pred)
                  (bvsgt y_pred y_pred)
                  (bvsge y_pred y_pred))))))
\end{lstlisting}
\end{comment}
\begin{lstlisting}[basicstyle={\ttfamily}]
(set-logic BV)
(synth-fun f ((x (_ BitVec 32))) (_ BitVec 32)
   ((y_term (_ BitVec 32)) (y_pred Bool))
   ((y_term (_ BitVec 32) ((Constant (_ BitVec 32))
                           (Variable (_ BitVec 32))
                           (bvnot y_term)
                           (bvand y_term y_term)
                           (bvor y_term y_term)
                           (bvneg y_term)
                           (bvadd y_term y_term)
                           (bvmul y_term y_term)
                           (bvudiv y_term y_term)
                           (bvurem y_term y_term)
                           (bvshl y_term y_term)
                           (bvlshr y_term y_term)
                           (ite y_pred y_term y_term)))
    (y_pred Bool (...
                  (bvult y_pred y_pred))))))
\end{lstlisting}
Constants may be specified
using either the hexidecimal format $\hexconst$
or the binary format $\binaryconst$.


\subsection{Strings}

Notice that there is no
officially adopted SMT-LIB standard for strings at the time
this document was written.
Thus, we adopt the proposed standard~\cite{}
as of \rem{fixed time}.


\section{Reserved Words}
\label{apx:reserved}

A \emph{reserved word} is any of the 
literals from Section~\ref{ssec:literals},
or any of the following keywords:

\section{Language Features of SMT LIB not Covered}

For the purpose of self-containment,
many of the essential language features of SMT LIB version 2.6
are redefined in this document.
However, others are omitted.
%We do not explicitly forbid the language features of SMT LIB .
We briefly mention other pertinent language features not mentioned in
this document.

Parametric datatype definitions

Recursive functions 

Match terms

Attribute annotations

\end{appendix}


\end{document}
