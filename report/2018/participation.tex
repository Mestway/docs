\section{Submitted Benchmarks and Solvers}
\label{sec:participation}

In addition to the benchmarks from the last year's competition,
we received over $100$ new benchmarks this year,
across various competition tracks, which we summarize below in \Cref{tbl:new-benchmarks}.

\begin{table}[!h]
	\def\arraystretch{1.1}
	\small
	\begin{center}
		\begin{tabular}{r||c|l}
			\textbf{Track} & \textbf{Benchmarks} 	& \textbf{Contributors} \\\hline \hline
			CLIA 		   & 	15 		   			& Kangjing Huang (Purdue University) \\
			General 	   & 	29 		   			& Qinheping Hu and Loris D' Antoni (University of Wisconsin-Madison) \\
			Invariants 	   & 	21 + 32    			& Saswat Padhi (UCLA) $\enskip$+$\enskip$ Kangjing Huang (Purdue University) \\
			PBE-Strings    & 	10 		   			& Woosuk Lee (University of Pennsylvania) \\
		\end{tabular}
	\end{center}
	\captionsetup{skip=0em}
	\caption{New benchmarks contributed to various tracks}
	\label{tbl:new-benchmarks}
\end{table}

Five solvers were submitted to this year's competition:
\begin{inlist}
	\item \cvcnew, an improved version of \cvc,
	\item \dryd, a solver specialized for conditional linear integer arithmetic,
	\item \eusolvernew, an improved version of \eusolver,
	\item \horndini, a solver specialized for constrained horn clauses (CHCs) and
	\item \lig, a solver specialized for invariant generation problems.
\end{inlist}
\Cref{tbl:solvers-authors} lists the submitted solvers along with their authors,
and \Cref{tbl:solvers-in-tracks} shows the tracks in which each solver participated.

\begin{table}[b]
	\small
	\begin{center}
		\scalebox{0.975}{
		\renewcommand{\arraystretch}{1.2}
		\begin{tabular}{r||l}
			\textbf{Solver} & \textbf{Authors} \\ \hline \hline
			\cvcnew 		& Andrew Reynolds (Univ. of Iowa), Haniel Barbosa (Univ. of Iowa), Andrez Notzli (Stanford), \\
							& Cesare Tinelli (Univ. of Iowa), and Clark Barrett (Stanford) \\[6pt]
			\dryd           & KangJing Huang (Purdue Univ.), Xiaokang Qiu (Purdue Univ.), Qi Tan (Nanjing Univ.), and \\
							& Yanjun Wang (Purdue Univ.) \\[6pt]
			\eusolvernew  	& Arjun Radhakrishna (Microsoft) and Abhishek Udupa (Microsoft) \\[6pt]
			\horndini  		& Deepak D’Souza (IISc, Bangalore), P. Ezudheen (IISc, Bangalore), P. Madhusudan (UIUC), \\
							& Pranav Garg (Amazon), Daniel Neider (MPI-SWS), and Shubham Ugare (IIT, Guwahati) \\[6pt]
			\lig            & Saswat Padhi (UCLA), Rahul Sharma (Microsoft Research), and Todd Millstein (UCLA) \\
		\end{tabular}}
	\end{center}
	\captionsetup{skip=0em}
	\caption{List of registered solvers}
	\label{tbl:solvers-authors}
\end{table}


The \cvcnew\ solver is based on an approach for program synthesis that is implemented inside an SMT solver~\cite{ReynoldsDKTB15}.
This approach extracts solution functions from unsatisfiability proofs of the negated form of synthesis conjectures,
and uses counterexample-guided techniques for quantifier instantiation (CEGQI) that make finding such proofs practically feasible.
\cvcnew\ also combines enumerative techniques, and symmetry breaking techniques~\cite{ReynoldsT17}. 

The \dryd\ solver combines enumerative and symbolic techniques.
It considers benchmarks in conditional linear integer arithmetic theory (LIA), and can therefore assume all have a solution in some pre-defined decision tree normal form.
It then tries to first get the correct height of a normal form decision tree, and then tries to synthesize a solution of that height.
It makes use of parallelization, using as many cores as are available, and of optimizations based on solutions of typical LIA SyGuS problems.

The \eusolvernew\ solver uses a divide-and-conquer strategy~\cite{AlurCAV15} to find different expressions that satisfy different subsets of the input space,
and then unifies them into a solution that works well for the entire space of inputs.
Subexpressions are typically found using enumeration techniques
and are then unified into the final solutions using decision tree learning~\cite{AlurRU17}.

\begin{wraptable}{r}{5cm}
	\setlength{\tabcolsep}{4pt}
	\def\arraystretch{1.2}
	\begin{center}
		\begin{tabular}{r||rrrrr}
			& \multicolumn{5}{c}{\textbf{Solver}} \\[8pt]
			\textbf{Track} & \rot{\cvcnew} & \rot{\dryd} & \rot{\eusolvernew} & \rot{\horndini} & \rot{\lig} \\ \hline \hline
			CLIA        & 1 & 1 & 1 & 0 & 0 \\
			INV         & 1 & 1 & 1 & 1 & 1 \\
			General     & 1 & 0 & 1 & 0 & 0 \\ 
			PBE-Strings & 1 & 0 & 1 & 0 & 0 \\ 
			PBE-BV      & 1 & 0 & 1 & 0 & 0 \\
		\end{tabular}
	\end{center}
	\captionsetup{skip=0em}
	\caption{Participating solvers}
	\label{tbl:solvers-in-tracks}
\end{wraptable}

The \horndini\ solver extends the classical IC3 decision-tree algorithm with
the Horn implication counterexamples (Horn-ICE) framework~\cite{EzudheenND0M18},
which extends the ICE-learning model.
The authors describe a decision-tree learning algorithm that learns from Horn-ICE samples,
works in polynomial time, and uses statistical heuristics to learn small trees that satisfy the samples.

The \lig\ solver~\cite{PadhiM17} for invariant synthesis extends the data-driven approach to inferring sufficient loop invariants
from a set of program states~\cite{PadhiSM16}.
Previous approaches to invariant synthesis were restricted to using a fixed set, or a fixed template for features,
e.g., ICE-DT~\cite{GNMR16} requires the shape of constraints (such as octagonal) to be fixed apriori.
Instead \lig\ starts with no initial features, and automatically learns features as necessary using program synthesis.
It reduces the problem of loop invariant inference to a series of precondition inference problems,
and uses a counterexample-guided inductive synthesis (CEGIS) loop to revise the current candidate.