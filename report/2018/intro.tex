\section{Introduction}
\label{sec:intro}

The \emph{Syntax-Guided Synthesis Competition} (\comp) is an annual competition aimed to provide
an objective platform for comparing different approaches for solving the \emph{Syntax-Guided Synthesis} (SyGuS) problem.
A SyGuS problem takes as input a logical specification $\varphi$ for what a synthesized function $f$ should compute,
and a grammar $G$ providing syntactic restrictions on the implementation for the function $f$ to be synthesized.
Formally, a solution to a SyGuS instance $(\varphi,G,f)$ is a function $f_{imp}$ that is expressible in the grammar $G$
such that the formula $\varphi[f/f_{imp}]$ obtained on replacing $f$ by $f_{imp}$ in the logical specification $\varphi$ is valid.
SyGuS instances are formulated in SyGuS-IF~\cite{RaghothamanU14}, a format built on top of SMT-LIB2~\cite{smtlib}.

We report here on the 5\textsuperscript{th} SyGuS competition that took place in July 2018,
in Oxford, UK as a satellite event of CAV'18 (The 30\textsuperscript{th} International Conference on Computer Aided Verification)
and SYNT'18 (The 7\textsuperscript{th} Workshop on Synthesis).
As in the previous competition, there were four tracks:
the general track, the conditional linear integer arithmetic (CLIA) track, the invariant synthesis (Inv) track,
and the programming by examples (PBE) track.
We assume most readers of this report are already familiar with the SyGuS problem and the \comp{} tracks,
and thus refer the unfamiliar reader to the report on last year's competition~\cite{SyGuSComp17}.

The report is organized as follows.
\begin{itemize}
    \item Section~\ref{sec:benchs} describes the participating benchmarks.
    \item Section~\ref{sec:solvers} lists the participating solvers and briefly describes the main idea behind their strategy.
    \item Section~\ref{sec:exp-set} provides details on the experimental setup.
    \item Section~\ref{sec:comp-results} gives an overview of the results per track.
    \item Section~\ref{sec:benchs-pres} provides details on the results, given from a single benchmark perspective.
    \item Section~\ref{sec:discussion} concludes the report with some key takeaway points.
\end{itemize}
 